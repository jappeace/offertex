\documentclass{scrartcl}
\usepackage[dutch]{babel}
\usepackage{fancyhdr}
\usepackage{wallpaper}
\usepackage{listings}

\begin{document}
\title{Handleiding offertex}
\date{\today}
\author{Jappie Klooster}
\maketitle
\section{Opzet}
Het programma heeft \emph{pdflatex} nodig om pdf bestanden te maken en
\emph{python 3.4} nodig om succesvol te kunnen worden uitgevoerd.

\LaTeX word gebruikt om van een text bestand een mooi getypeset bestand
te maken.

\section{Gebruik}
in de root directory voer het volgende commando uit in uw favorite terminal
emulator:
\begin{lstlisting}[language=bash]
	./program.py
\end{lstlisting}

Het programma stelt u dan een lijst met vragen. Het doet dit op basis
van de variablen (gemakeert met \$) gevonden in het offer.tex bestand.
Als alle vragen beantwoord zijn dan worden er een \$NAAM.tex en \$NAAM.pdf
gegenereerd in de out folder. Dus als alles goed is gegaan dan kun je
het pdf besand in 1 keer opsturen. Als er nog fouten in staan an
kun je eventueel handmatig het \$NAAM.tex bestand aanpassen en een
nieuw pdf bestand generereren met \emph{pdflatex}.

\subsection{Configuratie}
Het programma of de variable naam bestaat in de variables
folder, als dit het geval is dan kunnen er andere soorten menus worden
gegenereerd op basis van de overeenkomende naam (bijvoorbeeld in het
geval van .select en .options). De .options regels kunnen zelf ook variabelen
bevatten.

In de details folder worden de ``ingredienten'' van de verschillende buffetten 
omschreven. Dit is wederom in \LaTeX formaat.

\subsubsection{Aanpassen template}
Als u niet tevreden bent met het uiterlijk kunt u er voor
kiezen om het offer.tex bestand aan te passen. zorg ervoor dat
deze aanpassingen succesvol kunnen worden behandeld met \emph{pdflatex},
anders zal er geen pdf bestand gegenereerd worden.

om te testen of het werkt voer het volgende commando uit:

\begin{lstlisting}[language=bash]
	pdflatex offer.tex
\end{lstlisting}

als er geen errors of warnings worden weergeven (en er een pdf bestand
verschijnt, of deze word vernieuwd), dan zijn de wijzigingen goed.

\section{FAQ}
\subsection{Waarom word er gebruik gemaakt van \LaTeX}
Omdat latex geheel in ascii geschreven is, is het extreem makkelijk
om een latex bestand te gebruiken als template en daar variablen in
te vullen met behulp van een script taal (python in dit geval)

Een office pakket gebruikt doorgaans een binair formaat. Dit betekent
dat je bestanden alleen kan aanpassen via het office pakket. Wat
extreem onwenselijk is voor complexere opdrachten zoals deze.

Daarnaast zijn de gebruikte technologieen gratis en vrij aanpasbaar.
Dit betekent dat je niet afhankelijk bent van iemand als je dit
gebruikt (zoals wel het geval is bij ms office, namelijk microsoft).

\subsection{Terminal emulator, wat?}
Een programma waarin u commandos can geven aan de computer. Dit is
doorgaans de meest efficiente technologie te werken met text en bestanden.

op windows werkt een programma als cygwin het best. osx en linux hebben
al 1 voorgeinstaleerd.

\subsection{Geen grafische user interface?}
Nee, het maken van grafische user interfaces is extreem tijdrovend.
Daarnaast werken ze ineficienter. Je kan meer doen met text invoer.
\subsection{Waarom is het opensource?}
Omdat dit programma te moeilijk is om te verkopen, maar moeilijk
genoeg in gebruik voor het verkoop van ondersteuning. De perfecte
combinatie voor opensource.
\end {document}
